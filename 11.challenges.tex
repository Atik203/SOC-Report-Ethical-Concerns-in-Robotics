\section{Challenges}

The development and marketing of social robots that truly work across the board are still a long way off. Consumer robots that effectively perform tasks, like the robot vacuum cleaner Roomba, present an entirely different challenge, as noted by inventor Joe Jones (who is behind both Roomba and the newly launched weeding robot Tertill). This involves reworking human work, as "robots are better than people at some things and worse at others." This perspective emphasizes that "every application that you want to roboticize has to be re-imagined from the ground up" (Ackerman 2017).

\subsection{Technical Challenges}

The technical challenges are substantial, including the development of robots that can navigate a flight of stairs (Guardian 2017b) and learn from their environment. 

\subsection{Philosophical and Ethical Challenges}

In addition to technical hurdles, there are philosophical and ethical challenges. Sparrow and Sparrow argue that "for the foreseeable future, it will be wrong for us to create emotional care robots to look after the elderly" (2006, p. 156). This concern brings forth what Sharkey and Sharkey (2012) describe as contentious issues in robot-assisted care for older people. These include:

- Risks of reduced human contact
- Increased objectification and loss of control
- Reduced privacy and potential curtailment of personal freedom
- Deception and infantilization of elderly individuals
- Debates about when elderly people should have control instead of robots

These analyses serve as counter-narratives to the technologization of caring relationships. Parks (2010) employs various philosophical perspectives, including feminist cultural relativism, social justice, and an ethos based on the capabilities approach and Habermassian public discourse theory, to argue against the replacement of human care with robots. This development prompts us to reflect on what aspects of care are uniquely human.

The question of robot rights, once confined to science fiction, is now central to elite ethical and policy debates. Khan (2011) posits that a new ontological entity between ‘object’ and ‘agent’ must be constructed for robots, as humanization and anthropomorphism of social robots are prevalent across baseline research (Chanseau 2016; Karreman 2016). Robotic autonomy is closely related to the concept of rights (Rini 2017). As robots gain the ability to make their own decisions, it raises questions about how we can influence these decisions for human benefit. Calo (2015) even argues for expanding the legal rights of robots, a point that could lead to the creation of a new body of law governing robots. In 2016, the European Parliament's legal affairs committee approved a report calling for an AI and robotics Bill of Rights (European Parliament 2016). Mady Delvaux, the report's author, emphasized the need for a robust European legal framework to ensure that robots serve humans effectively (Guardian 2017a).

Dautenhahn (2007) suggests that human-robot interaction (HRI) is a double-edged sword. While robots can make excellent caregivers, humans can also provide care for robots, a dynamic that is less understood (Lipp 2016). For instance, Cho and Shin (2011) examined how children with autism cared for a toy robotic dinosaur, PLEO. This scenario illustrates the emotional attachments and sense of companionship that can develop, akin to the 'Tamagotchi effect' (Holzinger and Maurer, 1999), where individuals form bonds with technology, caring for it significantly.

\subsection{User Experience (UX)}

A new field of user experience (UX) for robots has emerged alongside the development of socially interactive robots. UX refers to “people’s feelings about using technology in a specific context” (Alenljung et al. 2017, p. 1). Research is growing on creating positive user experiences for social robots, focusing on key frameworks of quality human-robot interaction, usability, learnability, safety, and trustworthiness.

However, some robot developers lack sufficient understanding of appropriate methodologies, leading to “quick and dirty” evaluation methods with questionable validity and reliability (Alenljung et al. 2017, p. 2). Manufacturers of socially interactive robots must provide research-based guidance on effective user experience design. Notably, there has been little direct experience regarding the extended real-world use of social robots by users.

Chu et al. (2017) conducted an observational longitudinal study (2010—2014) in an Australian residential care facility to measure user engagement levels with social robots. Their main objective was to investigate how social robots enhanced the quality of diversion therapy services for people with dementia (PWD) by enriching sensory experiences and providing enjoyment. They focused on four indicators: approaching social robots, experiencing enjoyment with the robots, interaction with the robots, and interactivity among participants.

During this trial, two third-generation social robots, Sophie and Jack, were used. They demonstrated capabilities such as face recognition, singing, dancing, gestures, and emotional expression. The study observed a substantial increase in social engagement among PWD from 2013 to 2014, attributed to improved social capabilities and increased responsiveness to user interaction. The authors concluded that the quality of care provided by robots had significantly improved during this period.

\subsection{Education and Training}

Current and future technologies in social care practice, social work, and aged care are not adequately addressed in higher education institutions and training programs. Students often lack opportunities to develop critical awareness and skills in assistive technology, alternative living technology, or human-robot interaction. In Ireland, the educational awards standards of Coru (the professional regulator) and QQI (the educational regulator) do not reference the impact of these technologies, and few CPD (continuous professional development) courses exist for social professionals in this area.

To prepare social workers for a society that incorporates social robots, the following pedagogical methods may be effective:

\begin{itemize}
    \item Consideration of the main philosophical, political, and ethical questions
    \item Development of approaches to work with social robots
    \item Investigation of the potential effects of social robotics on at-risk clients
    \item Addressing social issues, work relations, and organizational administration in the care sector
    \item Acquisition of competencies, including designing and programming robots
    \item Teaching the necessary technical terminology to function in interdisciplinary teams
    \item Developing research skills related to social robotics
    \item Learning about essential information, data, and professional contacts in social robotics
    \item Discussing assumptions and attitudes towards social robots
    \item Gaining the ability to explain social robot issues to laypeople
\end{itemize}

These queries, along with other sociological robotics-related issues, could enhance our understanding and practice of care.

\subsection{Artificial Intelligence and Machine Learning}

Robotics heavily relies on artificial intelligence (AI) and machine learning (ML) technologies, which can be seen as the "brains" of robots. These technologies enable robots to make decisions, operate based on input, and modify their actions according to outcomes. One of the most intriguing tasks is endowing robots with decision-making capabilities comparable to humans. Recognizing the value of walkthroughs in both AI and ML, robots are becoming smarter and more efficient than ever.

\subsection{Navigation and Mobility}

Although robots have a degree of mobility, further challenges remain in designing robots that can reliably navigate complex environments. Advancements in robotic limbs will be crucial for enabling robots to climb stairs, navigate obstacles, and avoid falling. While achieving these capabilities may be a long way off, they will likely be essential for various applications.

\subsection{Cost and Accessibility}

The general public must recognize the pressing need for affordable and accessible robots. This means that robots should not be limited to big tech firms and research institutions; instead, economically viable and readily available robots should be made accessible to the consumer market.

\subsection{Public Perception}

Public perception plays a critical role in the adoption of robotics. Understanding how popular culture, media, and films shape societal views of robots is essential. It is equally important to articulate the beneficial and life-enhancing qualities that robotics can bring to individuals, communities, and society.

\subsection{The Future of Robotics}

The future of robotics holds great potential for advances that could revolutionize the field. As we address the challenges presented, we may find ourselves in a position where robotics serves not only engineers and scientists but also enhances human-machine relations, ultimately changing the world. \cite{christopherson,unknown-author-2024}
