\section{Autonomy and Responsibility in Robotics}

\subsection{Autonomy}
\begin{itemize}
    \item \textbf{Spectrum of Autonomy:} Robots can be placed along a spectrum of autonomy. At one end, we have teleoperated robots, which are controlled directly by humans. At the opposite end, we have fully autonomous robots capable of self-driving and independent decision-making.
    
    \item \textbf{Ethical Issues:} As robots gain the capability for independent decision-making, ethical concerns arise. For example, self-driving cars must make split-second decisions during emergencies. How can we ensure that these choices align with ethical standards?
\end{itemize}

\subsection{Responsibility}
\begin{itemize}
    \item \textbf{Accountability:} Responsibility involves being accountable for actions taken. In the case of self-operating robots, questions arise about who addresses their behavior. Is it the robot, its creator, or the operator?
    
    \item \textbf{Perceptions of Duty:} Duty can be perceived differently across age brackets. As robots become more anthropomorphized and autonomous, they may be seen as more capable of performing responsible actions, raising further ethical considerations.
\end{itemize}

\subsection{Legal and Ethical Issues}
\begin{itemize}
    \item \textbf{Liability Challenges:} Legal challenges emerge when a robotic entity functioning autonomously causes damage, such as in the case of a delivery drone crash. Determining liability can be convoluted—should responsibility lie with the company that coded the operating system or the robot’s AI system? Legal frameworks are evolving to address these complexities.
\end{itemize}

\subsection{The Balance between Autonomy and Control}
\begin{itemize}
    \item \textbf{Efficiency versus Accountability:} Increased autonomy often results in higher efficiency. For instance, autonomous drones can scout large areas more quickly than humans. However, this autonomy raises questions about accountability—who is responsible for their actions?
    
    \item \textbf{Ethical Decision-Making:} To what extent and in what manner should machines be granted decision-making authority? Finding a balance is crucial; we desire robots to function effectively while also adhering to moral guidelines.\cite{unknown-author-2024}
\end{itemize}
