\section{Ethical Concerns in Robotics}

\subsection{Human-Robot Interaction}
\begin{itemize}
    \item \textbf{Privacy:} Personal space robots collect a massive volume of data. This data must be protected and used benevolently to safeguard users' privacy.
    
    \item \textbf{Trust:} Transparency is crucial for fostering trust between people and robots. Users need to understand how robots make decisions to establish a reliable interaction.
    
    \item \textbf{Emotional Dependency:} As robots become more sophisticated, there is a potential risk of emotional dependencies developing. This may affect human relationships and mental health if users form attachments to robots.
\end{itemize}

\subsection{Bias in AI Algorithms}
\begin{itemize}
    \item \textbf{Fairness:} AI algorithms can unknowingly promote biases present in their training data, leading to discrimination against certain groups. Ensuring fairness in AI requires careful consideration of the training datasets used.
    
    \item \textbf{Discrimination:} It is essential to ensure that robotic decision-making processes do not reinforce societal inequalities. Addressing biases in algorithms is crucial to prevent discrimination in outcomes.
\end{itemize}

\subsection{Labor and Employment}
\begin{itemize}
    \item \textbf{Impact of Job Displacement:} Automation and robotics can displace workers across various sectors. It is important to support these workers in transitioning to new jobs in different areas of the industry.
    
    \item \textbf{Future Workforce:} Preparing the workforce for an AI- and robot-dominated future involves upskilling and educating individuals with new resources. This will help workers adapt to changes in the job market and ensure a smoother transition.\cite{haselagerjablonka-linabneybekey}
\end{itemize}
