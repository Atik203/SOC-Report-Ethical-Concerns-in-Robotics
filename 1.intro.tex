\section{Introduction}

Robotics, once a domain of science fiction, has become an integral part of modern society, transforming industries, healthcare, and everyday life. From autonomous vehicles and AI-driven assistants to surgical robots and industrial automation, the capabilities of robots are expanding rapidly. While these advancements offer significant benefits in efficiency, safety, and innovation, they also raise critical ethical concerns. As robots undertake more complex tasks traditionally performed by humans, issues surrounding privacy, accountability, safety, and potential societal disruption come to the forefront. 

This report explores the ethical implications of robotics, examining key concerns such as the impact on employment, the rights and treatment of autonomous machines, the challenges of AI decision-making, and the societal consequences of widespread automation. Rescue robotics is a relatively young discipline within field robotics, aiming to provide rescuers with the ability to sense and act from a distance in disaster areas . Disasters can result from environmental or man-made events, leading to fatalities, injuries, and significant economic breakdowns . Rescue robots enable operators to access harsh conditions that may be too dangerous or slow for humans to enter. They serve as remote sensing platforms, allowing for interaction with devastated environments \cite{adams2014, kochersberger2014,}. 

For instance, a rescue robot can visually examine and map the interior of a collapsed building, inspect damage , and quickly remove heavy rubble to facilitate victim extrication. Rapid access and intervention should lead to fewer lives lost, fewer injuries, and faster recovery \cite{murphy2014}. The first reported use of rescue robots at a disaster site was in 2001 when the Center for Robot-Assisted Search and Rescue deployed robots from the DARPA Tactical Mobile Robots program at the World Trade Center disaster . Since then, rescue robots have been utilized in various disasters, including mine accidents, earthquakes, nuclear disasters, and hurricanes, gaining widespread prominence. 

The need for such robots is expected to increase across all phases of the disaster life-cycle \cite{murphy2014}. Consequently, the terms “rescue robots” and “disaster robots” will be used interchangeably throughout this report. The types of robots employed in disasters include Unmanned Ground Vehicles (UGVs), which carry various sensors and can traverse unstructured terrains; Unmanned Aerial Vehicles (UAVs), providing aerial support; and Unmanned Marine Vehicles (UMVs), conducting underwater inspections. While most of these robots are human-controlled, semi-autonomous systems that reduce the need for low-level operator control are becoming more common \cite{delmerico2019, zuzanek2014}. 

Operations in disaster settings present ethical challenges due to the hazardous, chaotic conditions under which responders operate, compounded by limited time and resources. Decisions about where to concentrate rescue efforts, what risks to take, and whom to prioritize are morally burdensome \cite{gustavsson2020}. The consequences of these choices can affect victims, responders, and other stakeholders. While policies and guidelines exist to support responders \cite{world2015, international2012}, limited guidance is available for non-medical roles and ethically informed decision-making in specific disaster settings \cite{gustavsson2020}. 

The growing presence of rescue robots introduces additional ethical complexities, influenced by the type of robots used and their deployment contexts. Although ethical concerns regarding robots have received attention in industries, military applications, and healthcare \cite{lichoki2011}, the ethical issues in disaster settings remain underexplored \cite{harbers2017}. This report aims to focus timely ethical reflection on rescue robotics before their widespread use becomes commonplace through a scoping review of the relevant literature \cite{battistuzzi2021}.
